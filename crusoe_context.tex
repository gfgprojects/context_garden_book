L'obiettivo di questa parte \`e quello di ottenere un'equazione dinamica che possa essere utilizzata per analizzare un problema economico e di mettere in evidenza che gli aspetti di interesse sono tre: la determinazione dello stato stazionario, lo studio qualitativo e la soluzione analitica del modello.
\section{Il modello}
Si immagini un singolo soggetto che coltiva un appezzamento di terra. Indichiamo con $k$ la quantit\`a di semi che pianta. Il raccolto $y$ \`e una funzione crescente della quantit\`a di sementi:
\startformula
y=f(k).
\stopformula
Una volta ottenuto il raccolto, il soggetto decide di consumarne una quantit\`a $c$ e di 
accantonare la parte restante $y-c$ per la semina del periodo successivo.

Chiediamoci ora qual \`e la variazione della quantit\`a di sementi ($\dot{k}$). Essa sar\`a pari alla nuova quantità di semi ($y-c$) diminuita della quantità di semi di partenza ($k$):

\startformula
\dot{k}=y-c-k=f(k)-c-k
\stopformula

Il problema, dal punto di vista economico \`e quello di trovare come evolver\`a la situazione per quanto riguarda le variabili endogene. In questo semplice modello, si tratta di trovare il cammino futuro di $k$. La dinamica della produzione $y$ si ricava poi inserendo quella di $k$ nella funzione di produzione $f()$.

Non \`e sempre possibile ottenere la soluzione analitica per il cammino di $k$. Un obiettivo meno ambizioso \`e quello di dare una soluzione qualitativa. Uno ancora meno ambizioso \`e quello di vedere se esiste uno stato stazionario.
Iniziamo dal pi\`u semplice.

\section{Stato stazionario}
In questo caso basta imporre $\dot{k}=0$ ovvero risolvere l'equazione
\startformula f(k)-c-k=0.\stopformula 
Per trovare una soluzione, occorre adottare una forma funzionale per la funzione di produzione ed adottare una funzione per il consumo. 
Se ad esempio poniamo $y=k^{\alpha}$ si ha:
\placeformula[eq:c_param]$$k^\alpha -c-k=0$$
L'individuazione della funzione del consumo è un argomento molto studiato in macroeconomia che riprenderemo in seguito. 

A livello didattico potremmo trattare la $c$ come un parametro del modello, caso che analizzeremo nel paragrafo successivo, oppure assumere che il soggetto consumi una percentuale del suo reddito, come faremo qui di seguito. 

Assumiamo dunque che 
\startformula c=\gamma y=\gamma k^\alpha.\stopformula  
Con questa assunzione, la (\in[eq:c_param]) diventa:
\placeformula[eq:c_gy]$$\dot{k}=k^\alpha-\gamma k^\alpha -k.$$
La condizione di stazionariet\`a è dunque:
\startformula sk^\alpha-k=0\stopformula 
dove $s=1-\gamma$. Ora \`e possibile determinare esplicitamente lo stato stazionario:
\startformula 
k/k^{\alpha}=s \Rightarrow k^{1-\alpha}=s \Rightarrow k^*=s^{\frac{1}{1-\alpha}} 
\stopformula 
