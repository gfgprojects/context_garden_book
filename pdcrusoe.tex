L'obiettivo di questa parte è quello di ottenere un'equazione dinamica
che possa essere utilizzata per analizzare un problema economico e di
mettere in evidenza che gli aspetti di interesse sono tre: la
determinazione dello stato stazionario, lo studio qualitativo e la
soluzione analitica del modello.

\section[title={Il modello},reference={il-modello}]

Si immagini un singolo soggetto che coltiva un appezzamento di terra.
Indichiamo con $k$ la quantità di semi che pianta. Il raccolto $y$ è una
funzione crescente della quantità di sementi:
\startformula y=f(k). \stopformula Una volta ottenuto il raccolto, il
soggetto decide di consumarne una quantità $c$ e di accantonare la parte
restante $y-c$ per la semina del periodo successivo.

Chiediamoci ora qual è la variazione della quantità di sementi
($\dot{k}$). Essa sarà pari alla nuova quantità di semi ($y-c$)
diminuita della quantità di semi di partenza ($k$):
\startformula \dot{k}=y-c-k=f(k)-c-k \stopformula Il problema, dal punto
di vista economico è quello di trovare come evolverà la situazione per
quanto riguarda le variabili endogene. In questo semplice modello, si
tratta di trovare il cammino futuro di $k$. La dinamica della produzione
$y$ si ricava poi inserendo quella di $k$ nella funzione di produzione
$f()$.

Non è sempre possibile ottenere la soluzione analitica per il cammino di
$k$. Un obiettivo meno ambizioso è quello di dare una soluzione
qualitativa. Uno ancora meno ambizioso è quello di vedere se esiste uno
stato stazionario. Iniziamo dal più semplice.

\popbackground

\section[title={Stato stazionario},reference={stato-stazionario}]

In questo caso basta imporre $\dot{k}=0$ ovvero risolvere l'equazione
\startformula f(k)-c-k=0. \stopformula Per trovare una soluzione,
occorre adottare una forma funzionale per la funzione di produzione ed
adottare una funzione per il consumo. Se ad esempio poniamo
$y=k^{\alpha}$ si ha: 
\placeformula[eq:c_param] $$k^\alpha -c-k=0$$

L'individuazione della funzione del
consumo è un argomento molto studiato in macroeconomia che riprenderemo
in seguito.

A livello didattico potremmo trattare la $c$ come un parametro del
modello, caso che analizzeremo nel paragrafo successivo, oppure assumere
che il soggetto consumi una percentuale del suo reddito, come faremo qui
di seguito.

Assumiamo dunque che
\startformula c=\gamma y=\gamma k^\alpha. \stopformula Con questa
assunzione, la \in[eq:c_param] diventa:
\placeformula[eq:c_gy]$$\dot{k}=k^\alpha-\gamma k^\alpha -k.$$


La condizione di stazionarietà è dunque:
\startformula sk^\alpha-k=0 \stopformula dove $s=1-\gamma$. Ora è
possibile determinare esplicitamente lo stato stazionario:
\startformula k/k^{\alpha}=s \Rightarrow k^{1-\alpha}=s \Rightarrow k^*=s^{\frac{1}{1-\alpha}} \stopformula

\section[title={Soluzione
qualitativa},reference={soluzione-qualitativa}]

La soluzione qualitativa dipende dalla funzione del consumo. Continuiamo
per il momento con il caso del consumo proporzionale al reddito, per poi
tornare, a fini didaddici, al caso di consumo come parametro.

\subsection[title={$c=\gamma y$},reference={cgamma-y}]

Riprendiamo dunque l'equazione (\goto{{[}eq:c_gy{]}}[eq:c_gy]). Al fine
di analizzarla qualitativamente è importante determinare il segno di
$\dot{k}$ che può essere dedotto dalla sua equazione:
\startformula \dot{k}=(1-\gamma)k^\alpha-k. \stopformula Si procede
tracciando graficamente le funzioni rappresentate dai due addendi del
lato destro. Tracciamo dunque nello stesso grafico sia la funzione
$y=sk^{\alpha}$ e la funzione $y=k$. Le due funzioni dipendono da $k$,
per cui, avremo questa variabile nell'asse delle ascisse. La funzione
$y=k$ è dunque la retta che esce dall'origine e ha inclinazione di 45
gradi. $y=sk^{\alpha}$ parte dall'origine degli assi e la sua forma
dipende dal parametro $\alpha$. In particolare, è convessa verso il
basso se $\alpha<1$, ovvero nel caso di interesse.

Si noti che il punto di incontro delle due funzioni rappresenta lo stato
stazionario calcolato in precedenza ($k^*$).

Per $0\le\alpha\le 1$, quando $k<k^*$ si ha $sf(k)>k$, questo significa
che $\dot{k}>0$ ovvero che $k$ tende ad aumentare. Al contrario se
$k>k^*$, $k$ diminuisce. In conclusione $k$ converge a $k^*$.

La nostra conclusione qualitativa è dunque la seguente:

quando $0\le\alpha\le 1$, il sistema converge alla soluzione di stato
stazionario.

\subsection[title={$c$ come parametro},reference={c-come-parametro}]

Ritorniamo ora al caso in cui $c$ è un parametro \cite[alternative=authoryear,lefttext={si veda },righttext={ p. 404}][intriligator02].
Possiamo scrivere \startformula \dot{k}=(k^\alpha-k)-c. \stopformula
Come in precedenza, partiamo dalla forma di $k^\alpha-k$ quando
$0<\alpha<1$. Per $c$ sufficientemente basso, imponendo la
stazionarietà, si individuano due valori $k_l(c)<k_h(c)$ tali che

$\dot{k}>0$ se $k_l<k<k_h$ e $\dot{k}<0$ per gli altri valori.

Ne segue che $k_h$ è stabile mentre $k_l$ è instabile.

Conviene analizzare i casi estremi. Per $c=0$ si ha che $k_l=0$ ma
questo punto non è stabile e prima o poi si finirà in $k_h$. L'altro
caso estremo si ha quando il livello di $c$ è tale che $k_l=k_h$.
Indicheremo questo punto con $k_{gr}$ dove $gr$ sta per {\em golden
rule} in quanto il consumo è il massimo possibile. La cosa da notare è
che se si parte da un livello di capitale più alto di $k_{gr}$, si
converge a questo livello, ovvero $k_{gr}$ è attrattivo, mentre se ci si
sposta minimamente a sinistra, $k_{gr}$ è repulsivo. Un punto con le
caratteristiche appena descritte prende il nome di punto di sella. Per
il momento ci fermiamo alla constatazione dell'esistenza di simili
punti. Entreremo nei dettagli di questo tipo di stati stazionari quando
verrà esposto il modello di Ramsey.

L'analisi quantitativa di questa versione del modello ci ha segnalato
che:

ci possono essere diversi punti di stato stazionario, essi posso essere
stabili, instabili o di sella.

\section[title={Soluzione
quantitativa},reference={soluzione-quantitativa}]

Ritorniamo ora al caso in cui $c=\gamma y$. La soluzione quantitativa si
ottiene risolvendo l'equazione differenziale
\placeformula[eq:crusoe_c=gammay]$$ \dot{k}=sk^{\alpha}-k$$
L'equazione non è lineare,
ma è possibile ricondurla ad una lineare \cite[alternative=authoryear,lefttext={si veda },righttext={ pag. 500}][chiang84].

Poniamo $z=k^{1-\alpha}$. Facendo la derivata di quest'ultima rispetto
al tempo abbiamo
\startformula \dot{z}=(1-\alpha) k^{-\alpha}\dot{k}. \stopformula
Sostituendovi la
(\goto{{[}eq:crusoe_c=gammay{]}}[eq:crusoe_cux3dgammay]) abbiamo
\placeformula[eq:crusoe_c=gammayz]$$ \dot{z}=(1-\alpha) k^{-\alpha}(sk^{\alpha}-k)=(1-\alpha)s-(1-\alpha)k^{1-\alpha}=(1-\alpha)s-(1-\alpha)z.$$

La soluzione di questa
equazione è data dalla somma di due componenti: la soluzione
dell'equazione omogenea associata $z_g$ e la soluzione particolare $z_p$
: \startformula z=z_g+z_p \stopformula

L'equazione omogenea è \startformula \dot{z}=-(1-\alpha)z \stopformula
essa può essere scritta come
\startformula \frac{\dot{z}}{z}=-(1-\alpha) \Rightarrow \frac{d \ln{|z|}}{dt}=-(1-\alpha) \stopformula
Integrando in $t$
\startformula \int \frac{d \ln{|z|}}{dt} dt=\int -(1-\alpha) dt \stopformula
Ora non è difficile risolvere gli integrali
\startformula \ln{|z|}+c_1=-(1-\alpha)t+c_2 \stopformula si noti che $z$
ha lo stesso segno di $k$ e che è logico richiedere la non negatività
per questa variabile. Questo ci consente di togliere il valore assoluto.
Se facciamo l'esponenziale in ambo i lati e poniamo $c=c_2-c_1$
otteniamo \startformula z_g=e^{-(1-\alpha)t}e^{c} \stopformula

Per $z_p$ proviamo la soluzione stazionaria che si ottiene risolvendo
\startformula (1-\alpha)s-(1-\alpha)z=0 \stopformula da cui otteniamo
\startformula z_p=s \stopformula

Occorre verificare che questa sia effettivamente una soluzione della
(\goto{{[}eq:crusoe_c=gammayz{]}}[eq:crusoe_cux3dgammayz]). Sostituendo
$s$ al posto di $z$ nella
(\goto{{[}eq:crusoe_c=gammayz{]}}[eq:crusoe_cux3dgammayz]) si ha
$\dot z=0$. Derivando ora $z_p$ rispetto al tempo si ha
\startformula \dot{z_p}=0 \stopformula giungiamo dunque allo stesso
risultato. Abbiamo dunque verificato che la $z_p$ è una soluzione
particolare della
(\goto{{[}eq:crusoe_c=gammayz{]}}[eq:crusoe_cux3dgammayz]).

La soluzione cercata è pertanto
\startformula z=z_g+z_p=e^{-(1-\alpha)t}e^{c}+s \stopformula Abbiamo
ancora la costante arbitraria $c$ che possiamo determinare utilizzando
le condizioni iniziali ovvero imponendo che in $t=0$ si abbia $z=z_0$
\startformula z_0=e^{-(1-\alpha)0}e^{c}+s=e^{c}+s \stopformula da cui si
ottiene \startformula e^{c}=z_0-s \stopformula Sostituendo si ha
\startformula z=[z_0-s]e^{-(1-\alpha)t}+s \stopformula Per ottenere la
soluzione in termini di $k$ basta risostituire la definizione di $z$
\startformula k^{1-\alpha}=[k_0^{1-\alpha}-s]e^{-(1-\alpha)t}+s \stopformula
da cui elevando ambo i membri alla $\frac{1}{1-\alpha}$ si ottiene
\startformula k=\left([k_0^{1-\alpha}-s]e^{-(1-\alpha)t}+s\right)^{\frac{1}{1-\alpha}} \stopformula

Ora possiamo notare che per $\alpha<1$, si ha
$\lim_{t\rightarrow \infty}e^{-(1-\alpha)t}=0$, ovvero $k$ converge a
$k^*=s^{\frac{1}{1-\alpha}}$

\section[title={Input durevole},reference={input-durevole}]

La caratteristica del modello precedente è che abbiamo un input che ha
un ammortamento pari a 1.

Un modello più generale prevede un parametro in più per il tasso di
ammortamento $\delta$. 
\placeformula[eq:crusoe_delta]$$ \dot{k}=sk^\alpha-\delta k$$

ripercorrendo l'analisi fatta precedentemente, si ottiene
\startformula \dot{z}=(1-\alpha)s-\delta(1-\alpha)z;
\qquad
z_g=e^{-\delta(1-\alpha)t}e^{c};
\qquad
z_p=\frac{s}{\delta} \stopformula

e la soluzione è
\startformula k=\left(\left[k_0^{1-\alpha}-\frac{s}{\delta}\right]e^{-\delta(1-\alpha)t}+\frac{s}{\delta} \right)^{\frac{1}{1-\alpha}} \stopformula
Il comportamento qualitativo è lo stesso del caso precedente, ma lo
stato stazionario è ora
\startformula k^*=\left(\frac{s}{\delta}\right)^{\frac{1}{1-\alpha}}. \stopformula

Il codice seguente calcola e visualizza la soluzione per un valore
iniziale di $k$ pari a 20 e per i seguenti valori dei parametri:
$s=0.2$, $\delta=0.01$ e $\alpha=0.3$.

\starttyping
s<-0.2
delta<-0.01
alpha<-0.3
k_0<-20
t<-seq(0,20,by=0.1)
k=((k_0^(1-alpha)-s/delta)*exp(-(1-alpha)*t)+s/delta)^(1/(1-alpha))
plot(t,k,type="l")
\stoptyping

Il codice seguente seguente aggiunge alcune linee al precedente per
calcolare e visualizzare le soluzioni per diversi valori iniziali di
$k$.

\starttyping
#file crusoe_solution.R
s<-0.2
delta<-0.01
alpha<-0.3
k_0<-20
t<-seq(0,20,by=0.1)
k=((k_0^(1-alpha)-s/delta)*exp(-(1-alpha)*t)+s/delta)^(1/(1-alpha))
plot(t,k,type="l",ylim=c(10,120))
deltak_0<-10
for(z in 1:10){
k_0<-k_0+deltak_0
k=((k_0^(1-alpha)-s/delta)*exp(-(1-alpha)*t)+s/delta)^(1/(1-alpha))
lines(t,k)
}
\stoptyping

La figura \goto{1}[fig:crusoe] mostra il grafico ottenuto dal codice.

\placefigure[][fig:crusoe]{soluzioni per alcuni valori iniziali di
$k$.}{\externalfigure[../dottorato/computer/R/crusoe.pdf]}

{\bf Esercizio} verificare cosa accade per $\alpha\ge1$.
