
Questo documento ha lo scopo di introdurre alcuni modelli che costituiscono la base della teoria macroeconomica prevalente. 

Si intende in modo particolare mettere il lettore nelle condizioni di poter calcolare, sia analiticamente che computazionalmente le soluzioni dei modelli dinamici presentati. L'obiettivo, dunque \`e quello di far capire, per ogni modello, quali siano gli elementi che costituiscono la soluzione e come essa possa essere praticamente ottenuta.

\`E doveroso avvertire il lettore che la comprensione di quanto qui esposto non deve essere considerata come un punto di arrivo, ma come un punto di partenza. Questa esposizione, infatti difetta largamente nel trattare gli aspetti teorici che danno un'ampia visione delle problematiche da gestire in questo tipo di modellistica (come ad esempio quelle relative all'esistenza, unicit\`a e stabilit\`a della soluzione). La scelta di una via pratica \`e motivata dal fatto che un'esposizione teorica potrebbe disorientare chi per la prima volta si accosta agli argomenti trattati qualora il fine ultimo dell'analisi non sia stato ben compreso.

La speranza \`e che il materiale esposto riesca a ``far toccare con mano'' il fine ultimo dell'analisi. 
Si \`e ceracato dunque di riportare il pi\`u possibile i passaggi analitici e i listati relativi alle soluzioni numeriche. Tale cura non \`e riservata a risultati e tecniche che sono illustrate in altre fonti bibliografiche rinviando il lettore a queste ultime. 

Gli strumenti necessari per affrontare questo percorso sono sostanzialmente due. Dal punto di vista matematico, il lettore deve conoscere gli strumenti che sono comunemente oggetto di corsi di analisi matematica di base (con particolare riferimento alla derivazione di funzioni di pi\`u variabili). Dal punto di vista computazionale sarebbe opportuno che il lettore abbia delle nozioni introduttive del linguaggio utilizzato dal software R. Il codice riportato pu\`o infatti essere utilizzato in R.
L'attenzione nello stendere il codice \`e stata quella di evitare di inserire degli accorgimenti che, seppur utili, avrebbero comportato un notevole aumento del numero di righe, rendendo difficile la comprensione degli algoritmi a chi affronta queste metodologie per la prima volta.

